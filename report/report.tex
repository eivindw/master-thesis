\documentclass[a4paper,english]{report}

\usepackage[latin1]{inputenc}
\usepackage[T1]{fontenc}
\usepackage{fourier}
\usepackage{babel,textcomp}
\usepackage[pdftex]{graphicx}
\usepackage{listings}
\usepackage{hyperref}
\usepackage{varioref}
\usepackage{cite}
\usepackage[color]{uiosloforside}
\usepackage{tikz}
\usetikzlibrary{shapes,arrows}

\setlength\topmargin{0in}
\setlength\headheight{0in}
\setlength\headsep{0in}
\setlength\textheight{8.7in}
\setlength\textwidth{6.5in}
\setlength\oddsidemargin{0in}
\setlength\evensidemargin{0in}
\setlength\parindent{0.25in}
\setlength\parskip{0.15in}
\setlength\columnsep{0.25in}

\definecolor{dkgreen}{rgb}{0,0.6,0}
\definecolor{gray}{rgb}{0.5,0.5,0.5}
\definecolor{mauve}{rgb}{0.58,0,0.82}
\definecolor{matnat}{rgb}{0.004,0.47,0.44}

% "define" Scala
\lstdefinelanguage{scala}{
  morekeywords={abstract,case,catch,class,def,
    do,else,extends,false,final,finally,
    for,if,implicit,import,match,mixin,
    new,null,object,override,package,
    private,protected,requires,return,sealed,
    super,this,throw,trait,true,try,
    type,val,var,while,with,yield},
  otherkeywords={=>,<-,<\%,<:,>:,\#,@},
  sensitive=true,
  morecomment=[l]{//},
  morecomment=[n]{/*}{*/},
  morestring=[b]",
  morestring=[b]',
  morestring=[b]"""
}

\lstset{basicstyle=\footnotesize,
  numbers=none,
  numberstyle=\tiny\color{gray},
  keywordstyle=\color{blue},
  commentstyle=\color{dkgreen},
  stringstyle=\color{mauve},
  frame=single,
  showstringspaces=false,
  breaklines=true,
  breakatwhitespace=true,
  tabsize=3,
  language=scala
}

\hypersetup{pdfborder={0 0 0},colorlinks=true,linkcolor=blue,urlcolor=blue,citecolor=blue}

\title{Creating high performance DSLs in Scala}
\author{Eivind Barstad Waaler (\emph{eivindwa})}

\begin{document}
\uiosloforside[kind={Master thesis},boxcolor=matnat,textcolor=white]

\chapter{Abstract}

Scala is an increasingly popular programming language running on the
JVM (Java Virtual Machine). One of its pronounced goals is to simplify
the creation of DSLs (Domain-Specific Languages) with extensible
syntax and combination of Object-Oriented and Functional
Programming. This thesis examines these possibilities with regard to
performance. It is demonstrated how you can use Scala to create
efficient DSLs on the JVM, utilizing parallel and concurrent
mechanisms found in modern computers.

\tableofcontents

\listoftables

\listoffigures

\chapter{Preface}

Description of master-thesis, amount of work.

Short introduction to department (IFI) and group (OMS).

Thanks to supervisor++

\chapter{Introduction}

\section{Problem Description}

Detailed description of why and what. 

\section{Method}

This section describes the research methods applied in this thesis. In
\cite{dyb08} the concept ``Evidence-Based Software Engineering'' (EBSE)
is introduced. Although this article is mainly targeted towards
practicioners looking to support decision-making I feel that it gives
a good foundation for evaluating research methods related to Software
Engineering. The main steps of EBSE are as follows:

\begin{enumerate}
\item Convert a relevant problem or need for information into an
  answerable question.
\item Search the literature for the best available evidence to answer
  the question.
\item Critically appraise the evidence for its validity, impact, and
  applicability.
\item Integrate the appraised evidence with practical experience and
  the client's values and circumstances to make decisions about
  practice.
\item Evaluate performance in comparison with previous performance and
  seek ways to improve it.
\end{enumerate}

With this list as basis I ended up with the following steps for this
thesis:

\begin{enumerate}
\item Create answerable questions -- Make sure the questions to
  research are sound and possible to answer.
\item Search existing literature -- Find literature to provide direct
  answers to the questions, or to support assumptions needed to
  sustain experiments.
\item Critically appraise the literature evidence -- Make sure the
  chosen literature actually provided evidence for its claims.
\item Conduct experiments -- The major part of the thesis work is of
  course related to the experiments conducted. The first three steps
  are important to make sure the experiments are based on sound
  evidence and former knowledge.
\item Critical evaluation -- The experiments and results found must be
  carefully evaluated. By being critical to my own results I hope to
  improve the overall quality of the thesis deliverance.
\end{enumerate}

\chapter{Domain Specific Languages}

Description of DSL terminology and existing literature and research.

\chapter{Scala}

Test reference here \cite{hud96}.

Short general description of Scala.

More detailed description on DSL capabilities in Scala.

More detailed description on concurrent/parallel capabilities in Scala.

\section{Scala and Java}

The relationship between Scala and Java is illustrated in figure
\vref{fig:scalajava}.

\begin{figure}
  \begin{center}
  \tikzstyle{block} = [rectangle, draw, fill=blue!20, text width=5em, text centered, rounded corners, minimum height=4em]
  \tikzstyle{line} = [draw, -latex']
  \tikzstyle{cloud} = [draw, ellipse,fill=red!20, text width=5em, text centered, node distance=2.5cm, minimum height=2em]
  \begin{tikzpicture}[node distance = 2cm, auto]
    % Place nodes
    \node [cloud] (class) {Java Bytecode (.class)};
    \node [block, left of=class, above of=class] (scalacompile) {Scala Compile (scalac)};
    \node [cloud, left of=scalacompile] (scala) {Scala Source (.scala)};
    \node [block, right of=class, above of=class] (javacompile) {Java Compile (javac)};
    \node [cloud, right of=javacompile] (java) {Java Source (.java)};
    \node [block, below of=class] (jvm) {Java Virtual Machine};
    \node [block, below of=scala] (scalalib) {Scala API (.jar)};
    % Draw edges
    \path [line] (scalacompile) -- (class);
    \path [line] (javacompile) -- (class);
    \path [line] (class) -- (jvm);
    \path [line] (scala) -- (scalacompile);
    \path [line] (java) -- (javacompile);
    \path [dotted, line] (scalalib) -- (jvm);
  \end{tikzpicture}
  \end{center}
  \caption{Showing the relationship between Scala and
    Java.\label{fig:scalajava}}
\end{figure}

\chapter{Results}

\section{DSL for Image Processing}

\subsection{Description}

General description of DSL.

\subsection{Syntax}

Pure syntax examples.

\subsection{Parallel Processing}

With images larger than a certain size, the operations performed in
the image analysis are bound to be heavy in terms of the number of
computations that need to be executed. This opens up a demand for
exploiting the capabilities found in modern computers with regard to
parallelization and concurrency. In the following sections we look at
different mechanisms available from the Scala language to achieve
better performance through utilization of the hardware resources
available in the computer. The first two sections discuss mechanisms
for concurrent programming on a multi-core/-processor environment,
while the third section discuss a mechanism for parallelization using
the GPU (Graphics Processing Unit).

\subsubsection{Actors API}
\label{sec:actors}

Describe Actors API with example from DSL.

\subsubsection{Futures API}
\label{sec:futures}

Describe Futures API with example from DSL.

\subsubsection{OpenCL with ScalaCL}
\label{sec:opencl}

Brief description of OpenCL \cite{opencl} and ScalaCL \cite{scalacl}.

Description of DSL example using ScalaCL.

\subsection{Efficient Implementation Tips}

General tips/lessons learned:

\begin{itemize}
  \item Datastructures -- \texttt{Array} over \texttt{List}.
  \item Iterating -- \texttt{while-loops} over \texttt{for comprehensions}.
  \item Scala and primitives
\end{itemize}

\begin{table}[ht]
  \centering
  \begin{tabular}{c c c c}
    \hline\hline
    & find & addlast & delfirst \\ [0.5ex]
    \hline\hline
    find & OK & I & I \\
    addlast & - & I & I \\
    delfirst & - & - & I \\
    \hline\hline
  \end{tabular}
  \caption{A pure example table.\label{tab:ex1}}
\end{table}

\chapter{Summary}

Summary.. $\ddot\smile$

\bibliography{master}
\bibliographystyle{plain}

\end{document}
