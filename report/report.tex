\documentclass[a4paper,english]{report}

\usepackage[latin1]{inputenc}
\usepackage[T1]{fontenc}
\usepackage{fourier}
\usepackage{babel,textcomp}
\usepackage[pdftex]{graphicx}
\usepackage{listings}
\usepackage{hyperref}
\usepackage{varioref}
\usepackage{cite}
\usepackage[color]{uiosloforside}
\usepackage{tikz}
\usetikzlibrary{shapes,arrows}

\setlength\topmargin{0in}
\setlength\headheight{0in}
\setlength\headsep{0in}
\setlength\textheight{8.7in}
\setlength\textwidth{6.5in}
\setlength\oddsidemargin{0in}
\setlength\evensidemargin{0in}
\setlength\parindent{0.25in}
\setlength\parskip{0.15in}
\setlength\columnsep{0.25in}

\definecolor{dkgreen}{rgb}{0,0.6,0}
\definecolor{gray}{rgb}{0.5,0.5,0.5}
\definecolor{mauve}{rgb}{0.58,0,0.82}
\definecolor{matnat}{rgb}{0.004,0.47,0.44}

% "define" Scala
\lstdefinelanguage{scala}{
  morekeywords={abstract,case,catch,class,def,
    do,else,extends,false,final,finally,
    for,if,implicit,import,match,mixin,
    new,null,object,override,package,
    private,protected,requires,return,sealed,
    super,this,throw,trait,true,try,
    type,val,var,while,with,yield},
  otherkeywords={=>,<-,<\%,<:,>:,\#,@},
  sensitive=true,
  morecomment=[l]{//},
  morecomment=[n]{/*}{*/},
  morestring=[b]",
  morestring=[b]',
  morestring=[b]"""
}

\lstset{basicstyle=\footnotesize,
  numbers=none,
  numberstyle=\tiny\color{gray},
  keywordstyle=\color{blue},
  commentstyle=\color{dkgreen},
  stringstyle=\color{mauve},
  frame=single,
  showstringspaces=false,
  breaklines=true,
  breakatwhitespace=true,
  tabsize=3,
  language=scala
}

\hypersetup{pdfborder={0 0 0},colorlinks=true,linkcolor=blue,urlcolor=blue,citecolor=blue}

\title{Creating high performance DSLs in Scala}
\author{Eivind Barstad Waaler (\emph{eivindwa})}

\begin{document}
\uiosloforside[kind={Master thesis},boxcolor=matnat,textcolor=white]

\chapter{Abstract}

Scala is an increasingly popular programming language running on the
JVM (Java Virtual Machine). It was first released in 2003 and is now
available in version 2.7.5. Scala has a number of features that could
be useful when developing Domain-Specific Languages (DSLs). The goal
of this paper is to provide an overview and evaluate these features
with examples and discussion.

\tableofcontents

\listoftables

\listoffigures

\chapter{Preface}

Description of master-thesis, amount of work.

Short introduction to department (IFI) and group (OMS).

Thanks to supervisor++

\chapter{Introduction}

\section{Problem Description}

Detailed description of why and what. 

\section{Method}

Description of main method -- exploratory research.

Description of theory search and selection. Review of existing
literature.

TODO - look through material from INF5500 about methods and searching.

\chapter{Domain Specific Languages}

Description of DSL terminology and existing literature and research.

\chapter{Scala}

Test referance here \cite{hud96}.

Short general description of Scala.

More detailed description on DSL capabilities in Scala.

More detailed description on concurrent/parallel capabilities in Scala.

\section{Scala and Java}

The relationship between Scala and Java is illustrated in figure
\vref{fig:scalajava}.

\begin{figure}
  \begin{center}
  \tikzstyle{block} = [rectangle, draw, fill=blue!20, text width=5em, text centered, rounded corners, minimum height=4em]
  \tikzstyle{line} = [draw, -latex']
  \tikzstyle{cloud} = [draw, ellipse,fill=red!20, text width=5em, text centered, node distance=2.5cm, minimum height=2em]
  \begin{tikzpicture}[node distance = 2cm, auto]
    % Place nodes
    \node [cloud] (class) {Java Bytecode (.class)};
    \node [block, left of=class, above of=class] (scalacompile) {Scala Compile (scalac)};
    \node [cloud, left of=scalacompile] (scala) {Scala Source (.scala)};
    \node [block, right of=class, above of=class] (javacompile) {Java Compile (javac)};
    \node [cloud, right of=javacompile] (java) {Java Source (.java)};
    \node [block, below of=class] (jvm) {Java Virtual Machine};
    \node [block, below of=scala] (scalalib) {Scala API (.jar)};
    % Draw edges
    \path [line] (scalacompile) -- (class);
    \path [line] (javacompile) -- (class);
    \path [line] (class) -- (jvm);
    \path [line] (scala) -- (scalacompile);
    \path [line] (java) -- (javacompile);
    \path [dotted, line] (scalalib) -- (jvm);
  \end{tikzpicture}
  \end{center}
  \caption{Showing the relationship between Scala and
    Java.\label{fig:scalajava}}
\end{figure}

\chapter{Results}

\section{DSL for Image Processing}

\subsection{Description}

General description of DSL.

\subsection{Syntax}

Pure syntax examples.

\subsection{Parallel Processing}

General description of parallel computing on Image Processing.

\subsubsection{Futures API}

Describe Futures API with example from DSL.

\subsubsection{OpenCL with ScalaCL}

Brief description of OpenCL and ScalaCL.

Description of DSL example using ScalaCL.

\subsection{Efficient Implementation Tips}

General tips/lessons learned:

\begin{itemize}
  \item Datastructures -- \texttt{Array} over \texttt{List}.
  \item Iterating -- \texttt{while-loops} over \texttt{for comprehensions}.
  \item Scala and primitives
\end{itemize}

\chapter{Summary}

Summary.. $\ddot\smile$

\bibliography{master}
\bibliographystyle{plain}

\end{document}
