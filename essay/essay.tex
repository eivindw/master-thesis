\documentclass[a4paper,english,twocolumn]{article}

\usepackage[latin1]{inputenc}
\usepackage[T1]{fontenc}
\usepackage{fourier}
\usepackage{babel,textcomp}
\usepackage[pdftex]{graphicx}
\usepackage{listings}
\usepackage{color}
\usepackage{hyperref}
\usepackage{varioref}
\usepackage{cite}

\setlength\topmargin{0in}
\setlength\headheight{0in}
\setlength\headsep{0in}
\setlength\textheight{8.7in}
\setlength\textwidth{6.5in}
\setlength\oddsidemargin{0in}
\setlength\evensidemargin{0in}
\setlength\parindent{0.25in}
\setlength\parskip{0.15in}
\setlength\columnsep{0.25in}

\definecolor{dkgreen}{rgb}{0,0.6,0}
\definecolor{gray}{rgb}{0.5,0.5,0.5}
\definecolor{mauve}{rgb}{0.58,0,0.82}

% "define" Scala
\lstdefinelanguage{scala}{
  morekeywords={abstract,case,catch,class,def,
    do,else,extends,false,final,finally,
    for,if,implicit,import,match,mixin,
    new,null,object,override,package,
    private,protected,requires,return,sealed,
    super,this,throw,trait,true,try,
    type,val,var,while,with,yield},
  otherkeywords={=>,<-,<\%,<:,>:,\#,@},
  sensitive=true,
  morecomment=[l]{//},
  morecomment=[n]{/*}{*/},
  morestring=[b]",
  morestring=[b]',
  morestring=[b]"""
}

\lstset{basicstyle=\footnotesize,
  numbers=none,
  numberstyle=\tiny\color{gray},
  keywordstyle=\color{blue},
  commentstyle=\color{dkgreen},
  stringstyle=\color{mauve},
  frame=single,
  showstringspaces=false,
  breaklines=true,
  breakatwhitespace=true,
  tabsize=3,
  language=scala
}

\hypersetup{pdfborder={0 0 0},colorlinks=true,linkcolor=blue,urlcolor=blue,citecolor=blue}

\title{DSL Capabilities in Scala}

\author{Eivind Barstad Waaler (\emph{eivindwa})\\
  \url{eivindwa@student.matnat.uio.no}}

\begin{document}
\maketitle

\section*{Abstract}

Scala is an increasingly popular programming language running on the
JVM (Java Virtual Machine). It was first released in 2003 and is now
available in version 2.7.7. Scala has a number of features that could
be useful when developing Domain-Specific Languages (DSLs). The goal
of this paper is to provide an overview and evaluate these features
with examples and discussion.

\section{DSL Description and Terminology}

``Domain-specific languages (DSLs) are languages tailored to a
specific application domain. They offer substantial gains in
expressiveness and ease of use compared with general-purpose
programming languages.''\cite{mer05}

We have two main types of DSLs; internal and external DSLs. Internal
DSLs uses the host language so that it gets the feel of a particular
language. They are compiled and run together with the host
language. Internal DSLs are also referred to as embedded DSLs or
Fluent Interfaces. Eternal DSLs have custom syntax and typically
requires you to write a full parser to process them.

Scala has features to support development of both internal and
external DSLs. They will be described in the following sections.

\section{Internal DSL Features}

Internal DSLs are typically developed as highly specialized APIs
(libraries). They run directly in the host language using only
features present in the language. As such they can take advantage of
the compiler and tool support for the host language. The disadvantage
is that you have to rely only on the features and syntax found in the
host language.

The main use for internal DSLs will typically be libraries where you
want to control the usage to a specific syntax or programming style. A
good example is testing frameworks for behaviour driven development,
like RSpec (\url{http://rspec.info}) or Cucumber
(\url{http://cukes.info}).

\subsection{Type Inference and Implicits}

The Scala compiler will try to infer the types used. So if the type is
obvious there is no need to specify it:

\begin{lstlisting}
val i = 42 // Type Int is infered
val d = i + 2.1 // Type Double is infered
\end{lstlisting}

The language also supports implicit type conversions. So if an object
does not have the method being called on it, the compiler will try to
convert it to a type that contains the method:

\begin{lstlisting}
// Wrapper class for int
class MyInt(val i: Int) {
  def doubleIt = i * 2 // Method
}
// Implicit conversion
implicit def fromInt(i: Int) = new MyInt(i)
// Convert Int to MyInt and call method
4.doubleIt
\end{lstlisting}

\subsection{Method Names and Operator Notation}

Scala has two features regarding method names and syntax that are
interesting with regards to DSL development:

\begin{itemize}
\item Arbitrary method names - Scala methods can have special
  characters as names. As such Scala has full support for operator
  overloading, as operators are just regular methods with special
  names. For example \texttt{+}, \texttt{!=} and \texttt{<=}.
\item Operator notation - Methods in Scala can be called without the
  dot and parentheses, as shown in the examples below.
\end{itemize}

\begin{lstlisting}
abstract class Matrix {
  // Method names can be operators
  def +(other: Matrix) = add(other)
  // Regular method name
  def add(other: Matrix): Matrix
}
// Regular notation
val m3 = m1.+(m2)
val m3 = m1.add(m2)
// Operator notation
val m3 = m1 + m2
val m3 = m1 add m2
\end{lstlisting}

\subsection{The ``magic'' \texttt{apply} Method}

Scala uses the \texttt{apply} method to let classes and object define
functionality that appears to be native in the language. In the
following example the \texttt{apply} method is used to index the
\texttt{Matrix} class as if it was a native feature:

\begin{lstlisting}
class Matrix[T] {
  def apply(row: Int, col: Int): T = ...
}
val intMatrix = ... // Matrix[Int]
// Get Int at position 2, 4
val intVal = intMatrix(2, 4)
\end{lstlisting}

\subsection{Traits - Polymorphic Abilities}

Scala has support for virtual types (abstract type members) and
familiy polymorphism\cite{ode03}, mixin composition\cite{ode05} and
higher-order genericity\cite{moo08}. These are all powerful features
that support polymorphic embedding of DSLs\cite{hof08}.

This allows building DSLs that can have several different
interpretations as reusable components. It can also be used to
effectively combine different DSLs into new ones. The paper
``Polymorphic Embedding of DSLs''\cite{hof08} describes this in detail
with examples.

A use of traits can be to split functionality in several traits and
create classes/objects with only the methods that we need. In the
following example the \texttt{Matrix} class does not have any methods,
but instances of the class can be mixed in with traits containing
various methods:

\begin{lstlisting}
class Matrix {} // No methods

trait AvgMtx { this: Matrix =>
  def avg(): Matrix = { ... }
}

trait MorphMtx { this: Matrix =>
  def erode(): Matrix = { ... }
  def dilate(): Matrix = { ... }
}

// Matrix with only avg() method
val m1 = new Matrix with AvgMtx
// Matrix with erode() and dilate() methods
val m2 = new Matrix with MorphMtx
// Matrix with "all" methods
val m3 = new Matrix with AvgMtx with MorphMtx
\end{lstlisting}

\section{External DSL Features}

An external DSL is a separate language with its own syntax. The main
advantage with this is that the DSL is completely free in the way
syntax and structure is defined. The main disadvantage (compared to an
internal DSL) is that is will not get compiler and tool support from
the host language.

The main use for external DSLs is separate languages for specific
tasks. Examples are data-languages like XML and JSON (which are both
implemented as external DSLs in Scala).

\subsection{Combinator Parsing}

Scala includes a library for building parser combinators - functions
and operators defined in Scala that will serve as building blocks for
parsers. The mechanism is described in detail in chapter 31 in the
``Programming in Scala'' book\cite{ode08}.

Simply explained this allows a developer to specify a grammar with
actions directly in Scala, thus building support for an internal
DSL. The example below show a simple grammar for arithmetic
expressions:

\begin{lstlisting}
import scala.util.parsing.combinator._ 

// Simple arithmetic parser
class Arith extends JavaTokenParsers { 
  def expr: Parser[Any] =
    term~rep("+"~term | "-"~term) 
  def term: Parser[Any] =
    factor~rep("*"~factor | "/"~factor) 
  def factor: Parser[Any] =
    floatingPointNumber | "("~expr~")" 
}

// Testing the parser
val parser = new Arith
val input = "2 * (4 + 5)" // OK input
println(parser.parseAll(parser.expr, input))
val bad = "2 * ((4 + 5)" // Bad parenthesis
println(parser.parseAll(parser.expr, bad))
\end{lstlisting}

The example does not perform any actions, but still shows how the
mechanism works. The first example passes ok and the second one yields
an error as it is missing a parenthesis:

\begin{lstlisting}[language=tex]
parser.parseAll(parser.expr, bad)

[1.13] failure: `)' expected but `' found

2 * ((4 + 5)
            ^
\end{lstlisting}

A simplified version of the above example implemented with actions
could look as follows:

\begin{lstlisting}
import scala.util.parsing.combinator._

class Arith extends JavaTokenParsers {
  def expr: Parser[Double] =
    factor~"+"~factor ^^ 
      { case f1~"+"~f2 => f1 + f2 }
  def factor: Parser[Double] =
    floatingPointNumber ^^ (_.toDouble)
}

// Testing
val p = new Arith
val parse = p.parseAll(p.expr, _: String).get
println(parse("2 + 3")) // Output: 5.0
println(parse("2.5 + 3.1")) // Output: 5.6
\end{lstlisting}

In this example the grammar is simple enough to evaluate directly in
the actions. It would be easy to create a separate evaluation system
instead. However it is implemented this allows for using the
combinator parsing framework to embed an external DSL directly in the
language.

\section{Conclusion}

Scala has a wide range of features that are useful when creating a
DSL. It can support internal (embedded) DSLs because of concise
syntax, a powerful type system and polymorphic abilities. Scala can
also directly support the development of an external DSL with the
parser combinator framework.

\bibliography{master}
\bibliographystyle{plain}

\end{document}
